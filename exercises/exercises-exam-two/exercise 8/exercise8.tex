\documentclass[12pt]{article} 
\input{../custom}
\graphicspath{{figures/}}
\def\showcommentary{1}


\title{Exercise}
\author{}
\date{}


\begin{document}
\maketitle

%\subsection*{Instructions}
%\begin{itemize}
%\item \textbf{Don't look at the solution yet!} This is for your benefit.
%\item This exercise must be submitted within 48 hours of the lecture in which it was given. 
%\item As long as you do the exercise on time, you get full credit---your performance does not matter.
%\item Without looking at the solution, take 5 minutes to try to solve the exercise.
%\item Pre-assessment: Write down how correct you think your answer is, from 0 to 100\%.
%\item Post-assessment: Now, study the solution and give yourself a ``grade'' from 0 to 100\%.
%\item Submit your work on the course website, including the pre- and post- assessments.
%\end{itemize}

\subsection*{Exercise}
You need to sample from the conditional distribution of $X\mid X<c$, where $X\sim\N(0,1)$ and $c\in \R$. Assume:
\begin{itemize}
    \item you can generate $\Uniform(0,1)$ random variables, and 
    \item you can evaluate both the c.d.f.\ $F(x)$ and the inverse c.d.f.\ $F^{-1}(u)$ of the $\N(0,1)$ distribution.
\end{itemize}
How would you draw samples from $X\mid X<c$?


\newpage
\vfill
\rotatebox{180}{
\begin{minipage}[t][\textheight][t]{\textwidth}
\subsection*{Solution}\scriptsize
\subsubsection*{Approach 1 (Simple, but not great)}
To draw a sample $Z$ from the distribution of $X\mid X<c$,
\begin{enumerate}
    \item sample $U\sim \Uniform(0,1)$,
    \item set $X = F^{-1}(U)$,
    \item if $X \geq c$ then return to step 1 (reject), otherwise, output $Z = X$ as a sample (accept).
\end{enumerate}
Why does it work? By the inverse c.d.f.\ method, we know $X = F^{-1}(U)\sim\N(0,1)$. By the rejection principle, if we reject any samples $X$
such that $X\geq c$, then what remains has the conditional distribution given $X<c$.
This approach is not ideal, however, since the rejection rate may be very high, especially when $c\ll 0$.
\subsubsection*{Approach 2 (Better)}
To draw a sample $Z$ from the distribution of $X\mid X<c$,
\begin{enumerate}
    \item sample $U\sim \Uniform(0,1)$,
    \item set $V = F(c)U$, and 
    \item set $Z = F^{-1}(V)$.
\end{enumerate}
Why does this work? Note that in Approach 1, rejecting when $X\geq c$ is identical to rejecting when $U\geq F(c)$, and by the rejection
principle, we know that the distribution of the $U$'s that remain after rejection is $U\mid U<F(c)$, in other words, $\Uniform(0,F(c))$.  
But that means that the rejection step can be bypassed completely by just sampling $V\sim\Uniform(0,F(c))$ and setting $Z = F^{-1}(V)$!
And we can directly sample $V\sim\Uniform(0,F(c))$, by drawing $U\sim\Uniform(0,1)$ and setting $V = F(c)U$.
\end{minipage}}

\end{document}






