\documentclass[12pt]{article} 
\input{../custom}
\graphicspath{{figures/}}
\def\showcommentary{1}


\title{Exercise}
\author{}
\date{}


\begin{document}
\maketitle

%\subsection*{Instructions}
%\begin{itemize}
%\item \textbf{Don't look at the solution yet!} This is for your benefit.
%\item This exercise must be submitted within 48 hours of the lecture in which it was given. 
%\item As long as you do the exercise on time, you get full credit---your performance does not matter.
%\item Without looking at the solution, take 5 minutes to try to solve the exercise.
%\item Pre-assessment: Write down how correct you think your answer is, from 0 to 100\%.
%\item Post-assessment: Now, study the solution and give yourself a ``grade'' from 0 to 100\%.
%\item Submit your work on the course website, including the pre- and post- assessments.
%\end{itemize}

\subsection*{Exercise}
You need to sample from the distribution with p.d.f.\ 
$$p(x) \propto x^{a-1}\,\I(0<x<b)$$
where $a,b>0$.
Assume you can generate $\Uniform(0,1)$ random variables. 
How would you draw samples from $p(x)$?


\newpage
\vfill
\rotatebox{180}{
\begin{minipage}[t][\textheight][t]{\textwidth}
\subsection*{Solution}\scriptsize
If we can get the c.d.f.\ and invert it, we can use the inverse c.d.f.\ method.
First, let's find the normalizing constant of the p.d.f. For any $c>0$, 
\begin{align}\label{eqn}
    \int_0^c x^{a-1} d x = \frac{x^a}{a} \Big\vert_0^c = \frac{c^a}{a}.
\end{align}
since $a>0$.  In particular, $\int_0^b x^{a-1} d x = b^a/a$, so
$$ p(x) = \frac{a}{b^a} x^{a-1}\,\I(0<x<b). $$
Thus, for $c\in(0,b)$, the c.d.f.\ is
\begin{align*}
    F(c) &= \int_0^c p(x) d x \\
         &= \int_0^c \frac{a}{b^a} x^{a-1}\,\I(0<x<b) d x \\
         &= \frac{a}{b^a} \int_0^c x^{a-1} d x \\
         &= \frac{a}{b^a} \frac{c^a}{a} = (c/b)^a
\end{align*}
using Equation \ref{eqn} again.  To solve for $F^{-1}$, we set $u = F(x)$ for $u\in(0,1)$ and solve for $x$:
\begin{align*}
    & u = (x/b)^a \\
    & u^{1/a} = x/b \\
    & b u^{1/a} = x
\end{align*}
Thus, if $U\sim\Uniform(0,1)$ then $b U^{1/a} \sim p(x)$.
\end{minipage}}

\end{document}






